\documentclass{article}
\usepackage{graphicx} % Required for inserting images
\usepackage{amsmath}
\title{MAP Python}
\author{ }
\date{May 2024}

\begin{document}


T1) 1) Soit $\epsilon > 0$,
on considère les événements $A_{n}=(|\hat{\theta_{n}} - \theta |>\epsilon)$


$\hat{\theta_{n}} \in L^{2}$ car les $X_{i}$ sont dans $L^{2}$, de plus, puisqu'ils sont indépendants :
$$V(\hat{\theta_{n}}) = (\frac{1}{n})^{2}\sum_{i=1}^{n}V(X_{i}) = \frac{1}{n^{2}}\sum_{i=1}^{n}1 = \frac{1}{n}$$
$$P(A_{n}) =  P(|\hat{\theta_{n}} - \theta |>\epsilon) = P( \theta - \epsilon < \hat{\theta_{n}} < \theta + \epsilon) =
\frac{1}{\sqrt{2\pi V(\hat{\theta_{n}}) }}\displaystyle \int_{\theta - \epsilon}^{\theta + \epsilon} e^{-\frac{(x-\theta)^{2}}{2V(\hat{\theta_{n}})}} \, \mathrm{d}x   $$
Or, une intégration par parties permet d'obtenir l'inégalité, pour $\epsilon < |\theta|$ si $\theta \neq 0$ :
$$
 \frac{1}{\sqrt{2\pi V(\hat{\theta_{n}}) }}\displaystyle \int_{\theta - \epsilon}^{\theta + \epsilon} e^{-\frac{(x-\theta)^{2}}{2V(\hat{\theta_{n}})}} \, \mathrm{d}x = \frac{1}{\sqrt{2\pi V(\hat{\theta_{n}}) }}(\left[\frac{-V(\hat{\theta_{n}})}{x}e^{-\frac{(x-\theta)^{2}}{2V(\hat{\theta_{n}})}}
\right]_{\theta - \epsilon}^{\theta + \epsilon} - \displaystyle \int_{\theta - \epsilon}^{\theta + \epsilon}\frac{V(\hat{\theta_{n}})}{x^{2}}e^{-\frac{(x-\theta)^{2}}{2V(\hat{\theta_{n}})}}\mathrm{d}x$$ 
$$ \frac{1}{\sqrt{2\pi V(\hat{\theta_{n}}) }}\displaystyle \int_{\theta - \epsilon}^{\theta + \epsilon} e^{-\frac{(x-\theta)^{2}}{2V(\hat{\theta_{n}})}} \, \mathrm{d}x \leq \frac{1}{\sqrt{2\pi n}}) (\left[\frac{-1}{x}e^{-n\frac{(x-\theta)^{2}}{2}}
\right]_{\theta - \epsilon}^{\theta + \epsilon} \leq \frac{1}{\sqrt{2\pi n}} (\frac{1}{\theta - \epsilon}-\frac{1}{\theta + \epsilon})e^{-n\frac{\epsilon^{2}}{2}} $$
Et cette inégalité reste vraie pour $\theta = 0$


Par comparaison, $P(A_{n})$ est le terme général d'une série convergente, ce qui nous permet d'appliquer le lemme de Borel-Cantelli qui donne que $P(\limsup_{n}An ) = 0$, ce qui équivalent à ce que $\hat{\theta_{n}}$ converge presque sûrement vers $\theta$.
\vspace{13pt}

2) $\hat{\theta_{n}}$ suit une loi normale $\mathcal{N}(\theta,\,\frac{1}{n})$ comme combinaison linéaire de variables gaussiennes indépendantes, donc $\sqrt{n}(\hat{\theta_{n}} - \theta) = \frac{\hat{\theta_{n}} - E(\hat{\theta_{n}})}{\sigma(\hat{\theta_{n}})}$ suit une loi normale centrée réduite.
\vspace{13pt}

3) On cherche $K_{\alpha}$ tel que $P_{\theta = \theta_{0}}(1_{(\hat{\theta_{n}} < K_{\alpha})} = 1 ) = P_{\theta = \theta_{0}}(\hat{\theta_{n}} < K_{\alpha}) = \alpha$ i.e $$P_{\theta = \theta_{0}}(\sqrt{n}(\hat{\theta_{n}} - \theta) < \sqrt{n}(K_{\alpha} - \theta_{0}))$$ c'est-à-dire : $$ \Phi(\sqrt{n}(K_{\alpha} - \theta_{0}))= \alpha
$$ d'où $$K_{a} = \frac{1}{\sqrt{n}}\Phi(\alpha)^{-1} + \theta_{0}$$
\end{document}
